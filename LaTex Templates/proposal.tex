%% The first command in your LaTeX source must be the \documentclass comand. This is the generic manuscript mode required for submission and peer review.
%\documentclass[sigconf]{acmart} % two column
\documentclass[manuscript,authorversion,nonacm]{acmart} % one column

%\documentclass[sigconf,authorversion,nonacm]{acmart}

%% To ensure 100% compatibility, please check the white list of
%% approved LaTeX packages to be used with the Master Article Template at
%% https://www.acm.org/publications/taps/whitelist-of-latex-packages

% Packages here:
\usepackage{subcaption}
\usepackage{multirow}
\usepackage{colortbl}
\usepackage[table,xcdraw]{xcolor}
\usepackage[table]{xcolor}
\usepackage{balance}
\usepackage{soul}
\usepackage{graphicx}
\usepackage{enumitem}
\usepackage{wrapfig}
\usepackage{makecell}

% Beamer presentation requires \usepackage{colortbl} instead of \usepackage[table,xcdraw]{xcolor}
% \usepackage[normalem]{ulem}
% \useunder{\uline}{\ul}{}

% remove author addresses at bottom of title page
\makeatletter
\let\@authorsaddresses\@empty
\makeatother

% for block quotes
\usepackage{etoolbox}
% \AtBeginEnvironment{quote}{\par\singlespacing\small}

%%
%% \BibTeX command to typeset BibTeX logo in the docs
\AtBeginDocument{%
  \providecommand\BibTeX{{%
    \normalfont B\kern-0.5em{\scshape i\kern-0.25em b}\kern-0.8em\TeX}}}

\settopmatter{printacmref=false}

%%
%% end of the preamble, start of the body of the document source.

\begin{document}

%%
%% The ``title`` command has an optional parameter,
%% allowing the author to define a ``short title`` to be used in page headers.
\title[Proposal]{Proposal}

%%
%% The ``author`` command and its associated commands are used to define
%% the authors and their affiliations.

\author{Put your group names here (first and last)}
\affiliation{%
  \institution{CMPU 250}
  \city{Spring 2025}
  \country{Vassar College}
}

%% short names on header of each page
\renewcommand{\shortauthors}{Short author names (just last names)}

\maketitle

%%%%%%%%%%%%%%%%%%%%%%%%%%%%%%%%%%%%%%%%%%%%%


\section{Introduction}

This paragraph introduces the topic of the proposal.

This paragraph explains the motivation for the research questions related to this topic.

This is an example in-line citation: Prior work argues that ... \cite{andalibi2017sensitive}. \newline

This is an example of a citation with an author name: \citet{andalibi2017sensitive} argue that ... .


%%%%%%%%%%%%%%%              %%%%%%%%%%%%%%%

\subsection{Research Questions}

In this project, we aim to explore the following research questions:

\begin{itemize}
    \item [\textbf{(RQ1)}] What is foo?
    \item [\textbf{(RQ2)}] How is bar?
    \item [\textbf{(RQ3)}] How much XYZ are there?
\end{itemize}


%%%%%%%%%%%%%%%%%%%%%%%%%%%%%%%%%%%%%%%%%%%%%


\section{Data Description}

This is a description of the data your propose to use in your project. (Be sure your data is in your data folder in your project repo, as well.)

\subsection{Description of what you expect to find} 

This is a sketch of some analysis you plan to do with your data and what you expect to find.


% bibliography
\newpage
\bibliographystyle{ACM-Reference-Format}
\bibliography{reference}


% appendix (if necessary)
\newpage 
\appendix


\end{document}